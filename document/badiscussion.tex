\RequirePackage{etoolbox}
\csdef{input@path}{%
 {sty/}% cls, sty files
 {img/}% eps files
}%
\csgdef{bibdir}{bib/}% bst, bib files

\documentclass[ba]{imsart}
%
\pubyear{0000}
\volume{00}
\issue{0}
\doi{0000}
\firstpage{1}
\lastpage{1}


%
\usepackage{amsthm}
\usepackage{amsmath}
\usepackage{natbib}
\usepackage[colorlinks,citecolor=blue,urlcolor=blue,filecolor=blue,backref=page]{hyperref}
\usepackage{graphicx}
\usepackage[utf8]{inputenc}
\usepackage{enumerate}
\usepackage{verbatim}
\usepackage{booktabs}
\usepackage[ruled,vlined]{algorithm2e}
\usepackage{algpseudocode}
\usepackage{color}
\usepackage{dsfont}
\usepackage{float}

\startlocaldefs
% ** Local definitions **
\endlocaldefs



\begin{document}

%\maketitle
%\title{}
\date{April 2020}

\begin{frontmatter}
\title{Discussion of ``Unified Framework for De-Duplication and Population Size Estimation'' by Tancredi et al.}

%\title{\thanksref{T1}}
%\thankstext{T1}{<thanks text>}
\runtitle{}

\begin{aug}
\author{\fnms{Nianqiao}~\snm{Ju}\thanksref{addr1,t1} \ead[label=e1]{nju@g.harvard.edu}},
\author{\fnms{Niloy}~\snm{Biswas}\thanksref{addr1}},
\author{\fnms{Pierre E.}~\snm{Jacob} \thanksref{addr1}},
\author{\fnms{Gonzalo}~\snm{Mena} \thanksref{addr1,addr2}},
\author{\fnms{John}~\snm{O'Leary}\thanksref{addr1}}
\and
\author{\fnms{Emilia}~\snm{Pompe}\thanksref{addr3}}
%\author{\fnms{<firstname>} \snm{<surname>}\thanksref{}\ead[label=e1]{}}
%\and
%\author{\fnms{} \snm{}}
\runauthor{Ju et al.}

\address[addr1]{Department of Statistics, Harvard University 
}

\address[addr2]{Harvard Data Science Initiative}

\address[addr3]{Department of Statistics, University of Oxford}
\thankstext{t1}{\printead{e1}}
%\thankstext{<id>}{<text>}
\end{aug}


%% ** Keywords **
%\begin{keyword}%[class=MSC]
%\kwd{}
%\kwd[]{}
%\end{keyword}
\end{frontmatter}

We congratulate the authors on their important contribution to the record linkage literature, performing data de-duplication in the presence of population size uncertainty.
The authors' method involves a mix of discrete parameters, such as a latent population size $N$ and an unobserved partition over $\{1,\ldots,n\}$, and continuous ones like the vectors $\beta_0$, $\beta'$ and $\theta$.
Estimating this model entails significant computational challenges, resembling those associated with mixture models and Bayesian non-parametric methods. These arise as much in the design of a sampling algorithm as in the assessment of its convergence.

We would like to highlight a few tools that can be used to build confidence in MCMC results under such conditions. Many of these involve Markov Chain couplings,
as in \citet{valen_johnson_1996,johnson1998coupling},
and more recently in \citet{glynn2014exact,Nikooienejad2016,jacob2017unbiased,biswas2019estimating}.
These methods allow for the assessment and removal
of the impact of the starting distribution. They apply when
it is possible to run multiple chains that evolve  
marginally according to the proposed algorithm and jointly so that they meet after a random number of iterations. 

In the following, we describe how to generate chains
${X_t^{(1)}=(\eta^{(1)},  \beta_0^{(1)}
\beta'^{(1)}, \theta^{(1)},N^{(1)})}$ and 
$X_t^{(2)}=(\eta^{(2)},  \beta_0^{(2)},
\beta'^{(2)},\theta^{(2)},N^{(2)})$ which follow the Gibbs sampler of Section 5 of this article and which meet exactly at a random time.
A basic strategy for coupling Gibbs methods involves coupling each conditional update. The full conditional distribution of label indicators $\eta$ is a Multinomial distribution on $\{1,\ldots,n\}$ with the vector of probabilities computed as in (5.5) of the article.
To couple these updates, we compute such probabilities for both chains
and implement a maximal coupling to obtain two labels which will be identical with the maximal probability.
For the continuous variables $\beta_0$, $\beta'$ and $\theta$,
which are updated with Metropolis--Hasting steps,
we can employ maximal couplings of the Normal or Dirichlet proposal distributions, and use common Uniform draws to accept
or reject them.
Finally we can update $N$ with an exact Gibbs step, truncating $N$ to a very large integer, and implement a maximal coupling of this step.


However, an interesting difficulty
arises, reminiscent of the infamous label switching
issue \citep{stephens2000dealing}. Suppose that a first chain
has $\eta^{(1)} = (1,4,3,4,2)$ and the second $\eta^{(2)} = (4,3,2,3,1)$,
in a simple example with $n=5$.
These labels correspond to the same partition, 
and yet the $\eta$-components of the chains
are different and thus the chains cannot coincide.
Judging from our toy experiments, the associated meeting time
would be long.
This can be alleviated by an 
additional relabeling step, to be performed
after the update of the components of $\eta$.
A simple strategy, for example, is to relabel $\eta$ according to the order of the occurrences of new blocks, from component $1$ to $n$.
That is, both the labels $\eta^{(1)} = (1,4,3,4,2)$ 
and $\eta^{(2)} = (4,3,2,3,1)$  would
be relabelled $(1,2,3,2,4)$. 
This relabeling needs to be accompanied
by an adequate reshuffling of the associated
parameters, namely the $\beta'$-components in the notation of Section 5.
Other, more sophisticated relabeling strategies could
be devised, perhaps inspired by 
the literature on label-switching issues
in mixture models \citep{stephens2000dealing,marin2005bayesian,fruhwirth2011label}.

We will make some R scripts implementing a coupling of the proposed Gibbs sampler available at
\url{https://github.com/EmiliaPompe/discussion_unified_framework},
along with some simple numerical experiments on the synthetic dataset \texttt{RLdata500} from the \texttt{R} package \texttt{RecordLinkage} analysed in Section 6 of the paper.


\bibliographystyle{ba}
\bibliography{references_discussion}
\end{document}